\likechapter{ВВЕДЕНИЕ}
В нынешнее время технологии достигли уровня, когда автоматизированные системы используются повсеместно и тем не менее остаются не занятые ниши. На отечественном рынке медицинского оборудования очень малленький выбор устройств, которые могут измерить глубину аккомодации глаза (аккомодометров), а то небольшое количество видов устройств что есть(cм. рис.\ref{fig:aka-01}) не имеют электроприводов из-за чего движение внутренних механизмов осуществляется за счет мускульного усилия пациента, либо врача, так же на таких устройствах крайне не удобно выполнять тренировочные упражнения, которые достаточно важны для увеличения, либо для профилактики уменьшения глубины аккомодации.

Множество других недостатков по сравнению с автоматизированным устройством, где все измерения, запись параметров, создание журнала измерений, математики расчетов средних и их погрешностей, конспект результатов осмотра пациента и много другое делается быстрее, с приложением наименьших усилий врача и пациента. С использование электроники можно создать востребованное устройство, в этой работе представлен проект создания автоматизированной системы современного аккомодометра(далее аппарата) разработанный по заказу научно-производственной лабораторией ''Медоптика''.
\begin{figure}[ht]
    \centering
    \includegraphics[scale=0.5]{akkomodometr-aka-01.jpg}
    \caption{Аккомодометр АКА-01.}
    \label{fig:aka-01}
\end{figure}	