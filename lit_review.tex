\chapter{ЛИТЕРАТУРНЫЙ ОБЗОР}
\section{Введение}
Аппарат для тренировки и контроля аккомодации является простым прибором. При работе пациент наблюдает через монокулярную оптическую систему тест-объект. За счет его перемещения изменяется положение наблюдаемого изображения относительно глаза пациента. Разные тест-объекты, ориентированны на функционально различное назначение. 

Фиксируя крайние положения резко наблюдаемого объекта, можно проконтролировать аметропию  для дали и для близи, а также объем аккомодации. Плавное перемещение объекта в пределах объема аккомодации обеспечивает возможность тренировки механизма аккомодации глаза. Острота зрения для дали и для близи контролируется при использовании тест-объектов с таблицами различных оптотипов.

Аккомодометр позволяет исследовать астигматизм и определять главные меридианы астигматического глаза. Проводить контроль ночной миопии при пониженной яркости или освещенности тест-объекта. 

Все измерения осуществляются субъективным методом, поскольку основаны на оценке пациентом качества наблюдаемого изображения.

Аппарат рекомендуется для использования в офтальмологической практике при лечебных и оздоровительных мероприятиях. По назначению лечащего врача-офтальмолога аппарат может использоваться в домашних условиях, как правило, для проведения процедур по тренировке аккомодации, закрепляющих амбулаторное лечение. Длительность и периодичность воздействия устанавливаются в зависимости от медицинских показаний.

\section{Устройство и принцип действия}
Аппарат может использоваться для контроля объема аккомодации (как проксиметр), а также для определения аметропии, астигматизма и направления главных меридианов астигматического глаза (как оптиметр). 

Действие аппарата основано на монокулярном наблюдении слайда(или картинки дисплея), изображение которого может перемещаться относительно глаза пациента. На рис.\ref{fig:OpticSche} изображена оптическая схема аппарата. Наблюдаемым тест-обьектом может быть как дисплей 5 или освещенный дисплеем слайд 6.

\begin{figure}[ht]
	\centering
     \includegraphics[scale=0.7]{The_optical_scheme_of_the_device.png}
	\caption{Оптическая схема аппарата.}
	\label{fig:OpticSche}
\end{figure}

При использовании слайда, между экраном и слайдом помещается матовый светофильтр. Объектив 7 формирует изображение объекта. Это изображение наблюдается глазом 3, который располагается вблизи выходного отверстия 8 оптической системы. Компоненты 5 и 6 конструктивно образуют узел каретки 4, перемещаемый относительно остальной части оптической системы. 

Положение каретки контролируется по шкалам 1 и 2. Верхняя шкала 2 оцифрованным в сантиметрах и определяет расстояние $S$ между изображением и глазом. Если объект располагается в фокусе объектива (как это показано на рис. \ref{fig:OpticSche}), то изображение находится в бесконечности($ S=\infty $). При перемещении объекта влево (ближе к глазу) объектив формирует мнимое изображение, когда точка пересечения лучей, прошедших через объектив, находится правее объектива. Расстояние $S$ изменяется от максимального значения (свыше $100$ см) до минимальной величины (менее $10$ см).

Нижняя шкала 1 оцифрована в диоптриях и предназначена для определения аметропии глаза (А). Показания шкал связаны между собой соотношением $ A=\frac{-100}{S}$. Расположение объекта в фокусе  ($ S=\infty $) соответствует нулевому отсчету по диоптрийной шкале: $A=0$.

Чем ближе каретка к глазу, тем на меньшем расстоянии наблюдается изображение. Так, например, при $A=-2$ дптр изображение находится от глаза на расстоянии $S=50$ см, при $A =-4$ дптр на расстоянии $25$ см.
Положительные значения аметропии соответствуют смещению объекта правее фокуса. При этом объектив формирует действительное изображение, расположенное левее глаза.

\section{Режимы работы}
\subsection{Контроль аметропии}
Аметропия определяется в двух точках: в ближней и дальней, максимальная острота зрения соответствует моменту, когда главный фокус глаза располагается на сетчатке, а значит это положение, если оно крайнее, дальняя или ближняя точка. Ниже на рис.\ref{fig:TestObjC} приведено изображение используемое для тест-объектов.

\begin{figure}[ht]
	\centering
	\includegraphics[scale=1]{Test_Obj_Cycles.png}
	\caption{Расположение знаков в тест-объекте.}
	\label{fig:TestObjC}
\end{figure}
\subsection{Контроль объема аккомодации}
Объем аккомодации($\triangle A$ см. табл.\ref{tab:AccDuane}) --- это то расстояние между ближней и дальней точкой, в которой глаз четко видит изображение. Следовательно, находим положения ближней и дальней точек, и их разность равна объему аккомодации.
\begin{table}
\centering
\begin{tabular}{|c|c|c|c|}
\hline 
Возраст, лет &$\triangle A$,дптр & Возраст, лет & $\triangle A$,дптр \\ 
\hline 
10 & 12 - 14 & 40 & 3 - 8 \\ 
\hline 
16 & 10 - 14 & 45 & 2 - 6 \\ 
\hline 
20 & 9 - 13 & 50 & 1 – 3 \\ 
\hline 
25 & 8 - 12 & 55 & 0,75 – 1,75 \\ 
\hline 
30 & 6 - 10 & 60 & 0,5 – 1,5 \\ 
\hline 
35 & 5 - 9 & • & • \\ 
\hline 
\end{tabular} 
\caption{Возрастные нормы абсолютной аккомодации (по Дуане).}
\label{tab:AccDuane}
\end{table}
\subsection{Проведение тренировки аккомодации}
Тренировка механизма аккомодации глаза производится при перемещении слайда в пределах установленного объема аккомодации с периодическими попытками расширения его границ. Тренировку проводят поочередно каждым глазом. При тренировке плавно перемещают слайд от дальней границы резкого видения к ближней границе и обратно. Следует стремиться к расширению границ аккомодации: при близорукости – дальней границы, при дальнозоркости – ближней. Приближение слайда к границе аккомодации будет приводить к размытию изображения.  Продолжительности тренировки каждого глаза составляет 3 ... 7 мин. В профилактических целях тренировку можно проводить после работы, связанной со значительными зрительными нагрузками.
\subsection{Контроль остроты зрения}
Контроль остроты зрения проводится с помощью специальных слайдов см. рис.\ref{fig:TestObjEC}. Особенности построения оптической системы аппарата обеспечивают сохранение углового размера знака при перемещении слайда. За счет этого один и тот же слайд может использоваться для контроля зрения вдаль и вблизи.

\begin{figure}[ht]
    \centering
    \begin{subfigure}[b]{0.3\textwidth}
    \centering
        \includegraphics[scale=2]{Test_Obj_eye_control1.png}
        \caption{}
    \end{subfigure}
    \begin{subfigure}[b]{0.3\textwidth}
    \centering
        \includegraphics[scale=2]{Test_Obj_eye_control2.png}
        \caption{}
    \end{subfigure}
    \begin{subfigure}[b]{0.3\textwidth}
    \centering
        \includegraphics[scale=2]{Test_Obj_eye_control3.png}
        \caption{}
    \end{subfigure}
    \caption{ Тест-объекты для контроля остроты зрения.}
    \label{fig:TestObjEC}
\end{figure}

\section{Постановка задачи}
\subsection{Техническое задание}
Научно-производственная лаборатория "Медоптика" поставила следующее техническое задание:

1. Разработать модуль управления аппарата для тренировки и анализа аккомодации.

2. Модуль управления должен включать основной блок и блок интерфейсов.

3. Требования к основному блоку:

3.1. Основной блок модуля управления должен размещаться на подвижной каретке. Электронное управление основным блоком и его питание должно поступать от блока интерфейса.

3.2. Управление перемещением каретки должно осуществляться основным блоком. Управление перемещением каретки должно осуществляться тремя способами:

(1) с помощью рукоятки с угловым поворотом;

(2) посредством внешнего джойстика и клавиатуры, подключаемых к блоку интерфейсов;

(3) посредством команды, сформированной программой интерфейса.

Выбор способа управления перемещением выполняется с помощью программы интерфейса.

3.3. Основной блок должен содержать цветной светоизлучающий дисплей и слайд, располагаемый перед дисплеем; дисплей и слайд предназначены для формирования объекта наблюдения. При наличии слайда дисплей должен формировать одноцветные изображения для подсветки частей слайда; подсвеченные части слайда становятся объектом наблюдения. При отсутствии слайда дисплей должен формировать полноцветное изображение, которое становится объектом наблюдения в этом случае.

3.4. Основной блок должен передавать результаты измерений на монитор VGA, а хранение результатов должно осуществляться на извлекаемой флеш-памяти.

3.5. Основной блок должен при включении устанавливать каретку в позицию реперной точки.

3.6. Основной блок должен содержать узел распознавания слайда. Узел распознавания слайда должен определять отсутствие слайда и номер слайда при его наличии.

4. Требования к блоку интерфейсов:

4.1. Блок интерфейсов должен размещаться в стационарной задней части корпуса аппарата.

4.2. Блок интерфейсов должен содержать блок питания для создания всех необходимых вторичных напряжений для питания основного блока и блока интерфейсов.

5. Требования к конструкции модуля управления:

5.1. Между основным блоком и блоком интерфейсов связь должна обеспечиваться с помощью плоского шлейфа.

5.2. Основной блок должен иметь минимальные габариты, соизмеримые с применяемым дисплеем и габаритами каретки. Согласовывается с конструктором.

5.3. Конструкция модуля должна включать концевые контактные выключатели для электронной (программной) фиксации граничных положений каретки.

5.4. Конструкция аппарата должна содержать шаговый двигатель для перемещения каретки. Предоставляется Заказчиком (тип двигателя укажет конструктор). Шаговый двигатель расположен неподвижно в корпусе аппарата.

6. Требования по питанию:

6.1. Входное напряжение питания должно быть в диапазоне от 5 до 12 В для возможности применения стандартного адаптера на 220В. Стандартный адаптер (источник питания) должен размещаться в стационарной задней части корпуса аппарата.

6.2. Блок питания должен содержать защиту от короткого замыкания и от переполюсовки входного напряжения.

7. Требования к функциональной части:

7.1. Должно быть обеспечено выполнение алгоритмов измерения, тренировки и индикации результатов в соответствии с описанием работы Аппарата. Желательным является предоставление заказчику возможности для самостоятельного корректирования/развития алгоритмов работы Аппарата после завершения проекта.

7.2. Прошивка модуля управления должна обеспечивать управление перемещением каретки вперед и назад (или вверх и вниз) в прямой зависимости от органов управления (п.3.2).

7.3. При срабатывании концевых выключателей двигатель перемещения каретки должен останавливаться.

7.4. Прошивка модуля управления должна определять номер установленного слайда.

7.5. Прошивка должна обеспечивать засветку определенных частей слайда в зависимости от предварительно установленных центров для объектов на слайде. Имеется два типа слайдов:

- с набором из 10 … 15 мелких объектов в центральной зоне слайда; расположение центров одинаковое для всех слайдов этого типа;

- с одним большим объектом в центре слайда.
\subsection{Уточнение задания к дипломной работе}
В рамках дипломной работы, не ставится задача создать законченный аппарат, поскольку конструкционная разработка ведется отдельно, достаточно разработать электронику и создать операционную систему с целью демонстрации функциональных возможностей на макете и дальнейшей отладкой программного обеспечения. 
\subsection{Общая структура аппарата}
Система автоматизации должна управлять устройством, общие принципы которого описаны выше, так чтобы выполнение всех необходимых функций, так же описанных выше, было простым, удобным, быстрым, эффективным как для врача (далее администратора), так и пациента (далее клиента).

\begin{figure}[ht]
	\centering
     \includegraphics[scale=0.7]{function_shcem_of_sys_graf.jpg}
	\caption{Функциональная структура устройства.}
	\label{fig:graf:FunShcSys}
\end{figure}

 На блок-схеме (см.рис.\ref{fig:graf:FunShcSys}) показана функциональная структура устройства. Центральный обработчик - основной блок, выполняет  задачи транзакций между модулями, статистическую обработку данных, расчет графической оболочки, управление периферией и работу с внешним запоминающим устройством.

\begin{sidewaysfigure}[p]

\centerline{\epsfig{file=./images/apparats_shcem_of_sys_graf.jpg, scale=0.7}}

\caption{Аппаратная структура устройства.}
\label{fig:graf:AppShcSys}

\label{label_this_fig}

\end{sidewaysfigure}
На блок-схеме(см.рис.\ref{fig:graf:AppShcSys}) показана аппаратная структура устройства.
Далее в следующих главах мы выберем компонентную базу, подробно изучим элементы аппаратной структуры устройства, спроектируем электрическую схему и перейдем к операционной системе.