\likechapter{Глоссарий}
\begin{description}
\item[Аметропия] --- это изменение преломляющей способности человеческого глаза, следствием которого является то, что задний фокус глаза не попадает на сетчатку при расслаблении аккомодационной мышцы.\cite{b_1}

\item[Аккомодация] --- приспособление органа либо организма в целом к изменению внешних условий (значение близко к термину «адаптация»).\cite{b_2}

\item[Аккомодометр][ ккомод(ация)+ греч. metreo измерять, определят ] --- прибор для исследования аккомодации глаза.\cite{b_3}
\item[Удерживающий момент] - момент необходимый статору для блокировки ротора, когда шаговый двигатель находится под напряжением, но не вращается. Как правило, крутящий момент шагового двигателя на низкой скорости близок к удерживающему моменту. Так как выходной крутящий момент уменьшается с увеличением скорости, выходная мощность также изменяется, и удерживающий момент становится одним из наиболее важных параметров шагового двигателя.\cite{s_3}
\end{description}